\documentclass[tikz]{standalone}
\usepackage{tikz}
\usetikzlibrary{arrows,shapes,shapes.multipart,matrix,positioning,shadows,calc}

\makeatletter
\pgfarrowsdeclare{crow's foot}{crow's foot}
{
	\pgfarrowsleftextend{+-.5\pgflinewidth}%
		\pgfarrowsrightextend{+.5\pgflinewidth}%
}
{
	\pgfutil@tempdima=0.6pt%
		%\advance\pgfutil@tempdima by.25\pgflinewidth%
		\pgfsetdash{}{+0pt}%
		\pgfsetmiterjoin%
		\pgfpathmoveto{\pgfqpoint{0pt}{-9\pgfutil@tempdima}}%
		\pgfpathlineto{\pgfqpoint{-13\pgfutil@tempdima}{0pt}}%
		\pgfpathlineto{\pgfqpoint{0pt}{9\pgfutil@tempdima}}%
		\pgfpathmoveto{\pgfqpoint{0\pgfutil@tempdima}{0\pgfutil@tempdima}}%
		\pgfpathmoveto{\pgfqpoint{-8pt}{-6pt}}% 
		\pgfpathlineto{\pgfqpoint{-8pt}{-6pt}}%  
		\pgfpathlineto{\pgfqpoint{-8pt}{6pt}}% 
		\pgfusepathqstroke%
}

\pgfarrowsdeclare{omany}{omany}
{
	\pgfarrowsleftextend{+-.5\pgflinewidth}%
		\pgfarrowsrightextend{+.5\pgflinewidth}%
}
{
	\pgfutil@tempdima=0.6pt%
		%\advance\pgfutil@tempdima by.25\pgflinewidth%
		\pgfsetdash{}{+0pt}%
		\pgfsetmiterjoin%
		\pgfpathmoveto{\pgfqpoint{0pt}{-9\pgfutil@tempdima}}%
		\pgfpathlineto{\pgfqpoint{-13\pgfutil@tempdima}{0pt}}%
		\pgfpathlineto{\pgfqpoint{0pt}{9\pgfutil@tempdima}}%
		\pgfpathmoveto{\pgfqpoint{0\pgfutil@tempdima}{0\pgfutil@tempdima}}%  
		\pgfpathmoveto{\pgfqpoint{0\pgfutil@tempdima}{0\pgfutil@tempdima}}%
		\pgfpathmoveto{\pgfqpoint{-6pt}{-6pt}}% 
		\pgfpathcircle{\pgfpoint{-11.5pt}{0}} {3.5pt}
	\pgfusepathqstroke%
}

\pgfarrowsdeclare{one}{one}
{
	\pgfarrowsleftextend{+-.5\pgflinewidth}%
		\pgfarrowsrightextend{+.5\pgflinewidth}%
}
{
	\pgfutil@tempdima=0.6pt%
		%\advance\pgfutil@tempdima by.25\pgflinewidth%
		\pgfsetdash{}{+0pt}%
		\pgfsetmiterjoin%
		\pgfpathmoveto{\pgfqpoint{0\pgfutil@tempdima}{0\pgfutil@tempdima}}%
		\pgfpathmoveto{\pgfqpoint{-6pt}{-6pt}}% 
		\pgfpathlineto{\pgfqpoint{-6pt}{-6pt}}%  
		\pgfpathlineto{\pgfqpoint{-6pt}{6pt}}% 
		\pgfpathmoveto{\pgfqpoint{0\pgfutil@tempdima}{0\pgfutil@tempdima}}%
		\pgfpathmoveto{\pgfqpoint{-8pt}{-6pt}}% 
		\pgfpathlineto{\pgfqpoint{-8pt}{-6pt}}%  
		\pgfpathlineto{\pgfqpoint{-8pt}{6pt}}%    
		\pgfusepathqstroke%
}

\tikzset{
	entity/.code={
		\tikzset{
			label=above:#1,
			name=#1,
			inner sep=0pt,
			every entity/.try,
			fill=white  
		}%
		\def\entityname{#1}%
	},
		entity anchor/.style={matrix anchor=#1.center},
		every entity/.style={
			draw,
		},
		every property/.style={
			inner xsep=0.25cm, inner ysep=0.15cm, anchor=west, text width=1in
		},
		zig zag to/.style={
			to path={(\tikztostart) -| ($(\tikztostart)!#1!(\tikztotarget)$) |- (\tikztotarget) \tikztonodes}
		},
		zig zag to/.default=0.5,
		one to one/.style={
			one-one, zig zag to
		},    
		one to many/.style={
			one-crow's foot, zig zag to,
		},
		one to omany/.style={
			one-omany, zig zag to
		},      
		many to one/.style={
			crow's foot-one, zig zag to
		},
		many to many/.style={
			crow's foot-crow's foot, zig zag to
		}  
}

\def\pk#1{\node[name=\entityname-#1, every property/.try]{#1};\node[name=\entityname-#1, every property/.try, red, text width=1in, align=right]{(PK)};\\}
\def\fk#1{\node[name=\entityname-#1, every property/.try]{#1};\node[name=\entityname-#1, every property/.try, red, text width=1in, align=right]{(FK)};\\}
\def\property#1{\node[name=\entityname-#1, every property/.try]{#1};}
\def\properties{\begingroup\catcode`\_=11\relax\processproperties}
\def\processproperties#1{\endgroup%
	\def\propertycode{}%
		\foreach \p in {#1}{%
			\expandafter\expandafter\expandafter\gdef\expandafter\expandafter\expandafter\propertycode%
				\expandafter\expandafter\expandafter{\expandafter\propertycode\expandafter\property\expandafter{\p}\\}%
		}%
	\propertycode%
}

\begin{document}
	\begin{tikzpicture}[node distance=3cm]
		\matrix [entity=User] {
			\pk{id}
			\properties{
				name,
				password
			}
		};

		\matrix [entity=UserProject, below right=of User-id, entity anchor=UserProject-uid]  {
			\pk{id}
			\fk{uid}
			\fk{pid}
			\fk{rid}
		};

		\matrix [entity=Project, right=of UserProject-pid, entity anchor=Project-id] {
			\pk{id}
			\properties{
				name
			}
		};

		\matrix [entity=Role, below=of Project-id, entity anchor=Role-id] {
			\pk{id}
			\properties{
				name
			}
		};

		%\matrix [entity=Bug, above right=of Project-id, entity anchor=Bug-id] {
			%\pk{id}
			%\fk{projectId}
			%\properties{
				%name
			%}
		%};

		\matrix [entity=Release, right=of Project-id, entity anchor=Release-projectId] {
			\pk{id}
			\fk{projectId}
			\properties{
				version,
				startDate,
				endDate
			}
		};

		\matrix [entity=UserStory, above left=of Release-id, entity anchor=UserStory-releaseId] {
			\pk{id}
			\fk{statusId}
			\fk{ownerId}
			\fk{releaseId}
			\properties{
				name,
				description,
				points
			}
		};

		\matrix [entity=Task, right=of UserStory-id, entity anchor=Task-name] {
			\pk{id}
			\fk{statusId}
			\fk{assignedId}
			\fk{userStoryId}
			\properties{
				name,
				description
			}
		};

		\matrix [entity=Status, above left=of UserStory-id, entity anchor=Status-id] {
			\pk{id}
			\properties{
				name
			}
		};

		\matrix [entity=UserStoryComment, below right=of Task-id, entity anchor=UserStoryComment-id] {
			\pk{id}
			\fk{authorId}
			\fk{userStoryId}
			\properties{
				comments,
				createdOn
			}
		};


		\draw [one to omany] (User-id) to node[above]{} (UserProject-uid);
		\draw [one to omany] (Role-id) to node[above]{} (UserProject-rid);
		\draw [one to omany] (Project-id) to node[above]{} (UserProject-pid);
		%\draw [one to omany] (Project-id) to node[above]{} (Bug-projectId);
		\draw [one to omany] (Project-id) to node[above]{} (Release-projectId);
		\draw [one to omany] (Release-id) to node[above]{} (UserStory-releaseId);
		\draw [one to omany] (UserStory-id) to node[above]{} (Task-userStoryId);
		\draw [one to many] (Status-id) to node[above]{} (Task-statusId);
		\draw [one to many] (Status-id) to node[above]{} (UserStory-statusId);
		\draw [many to one] (UserStory-ownerId) to node[above]{} (User-id);
		%\draw [many to one] (Task-assignedId) to node[above]{} (User-id);
		\draw [one to omany] (UserStory) to node[above]{} (UserStoryComment-userStoryId);
	\end{tikzpicture}
\end{document}
